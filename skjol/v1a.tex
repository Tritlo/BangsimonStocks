\documentclass[a4,12pt]{article}
\usepackage[utf8]{inputenc}
\usepackage[T1]{fontenc}
\usepackage{a4,graphics,amsmath,amsfonts,amsbsy,amssymb,amsthm}
\usepackage{graphicx}
\usepackage{hyperref}
\usepackage{float}
\usepackage{listings}
\usepackage{enumerate}
\usepackage{comment}

\setlength{\parskip}{8pt plus 1pt minus 1pt}
%Verdur ad vera her, sumir pakkar dependa a thetta.
\usepackage[icelandic]{babel}

%viljum ekki númeraða kafla á dæmum

\newcommand{\nonums}{\setcounter{secnumdepth}{-1}}

\newcommand{\benum}{\begin{enumerate}[(i)]}
\newcommand{\aenum}{\begin{enumerate}[(a)]}
\newcommand{\eenum}{\end{enumerate}}


\nonums

\author{
Finnur Jónasson\\ fij2@hi.is,\\
Matthías Páll Gissurarson\\ mpg3@hi.is,\\
Ragnheiður Björk Halldórsdóttir\\ rbh1@hi.is,\\
Sólrún Halla Einarsdóttir\\ she5@hi.is
}
\title{Hópverkefni 1: Ábatinn\\
Fyrsti hluti}
\begin{document}
\maketitle

\section{Notendasögur}

\begin{enumerate}[]
\item Titill: Sækja söguleg gögn um hlutabréf fyrirtækis.\\
Lýsing: Forritið á að geta sótt söguleg gögn um hlutabréf fyrirtækis sjálfvirkt af netinu.

\item Titill: Sýnt gröf um fyrirtæki\\
Lýsing: Forritið á að geta sýnt ýmis gröf útfrá gögnunum um fyrirtækið

\item Titill: Sýna fréttir tengdar fyrirtæki\\
Lýsing: Forrit á að geta sýnt fréttir tengdar fyrirtæki.

\item Titill: Sýna fréttir\\
Lýsing: Forrit á að geta sýnt helstu fréttir af fjármálamarkaðinum.

\item Titill: Framkvæma tæknigreiningu\\
Lýsing: Forritið á að geta framkvæmt einfalda tæknigreiningu útfrá sögulegum gögnum.

\item Titill: Sækja söguleg gögn um vísitölur\\
Lýsing: Forritið á að geta sótt söguleg gögn um vísitölur sjálfvirkt af netinu.

\item Titill: Sýnt gröf um vísitölur\\
Lýsing: Forritið á að geta sýnt ýmis gröf útfrá gögnunum um vísitölur .

\item Titill: Sýna fréttir tengdar vísitölu \\
Lýsing: Forrit á að geta sýnt fréttir tengdar vísitölu 

\item Titill: Sýnt tæknigreiningar gröf\\
Lýsing: Forritið á að geta sýnt gröf útfrá tæknigreiningu á gögnum

\item Titill: Aðstoð við ákvarðanatöku\\
Lýsing: Forritið á að aðstoða við ákvarðanatöku með tæknigreiningunni

\item Titill: Sækja gögn um verslunarvörur\\
Lýsing: Forritið á að geta sótt söguleg gögn um verð verslunarvara sjálfvirkt af netinu.

\item Titill: Sýnt gröf um verslunarvörur\\
Lýsing: Forritið á að geta sýnt ýmis gröf út frá gögnum um verð á verslunarvörum

\item Titill: Samanburður á upplýsingum\\
Lýsing: Forrit á að geta borið saman upplýsingar um mismunani fyrirtæki eða verslunarvörur.

\item Titill: Skipting eftir löndum\\
Lýsing: Forritið á að geta skipt upp fyrirtækjum eftir í hvaða landi það er skráð.

\item Titill: Sækja gögn um gengi\\
Lýsing: Hægt á að vera að sækja söguleg gögn um gengi milli mismunandi gjaldmiðla

\item Titill: Sýna gröf um gengi\\
Lýsing: Hægt á að vera að sýna gröf um þróun gengis milli mismunandi gjaldmiðla.

\item Titill: Upphafssýn\\
Lýsing: Við opnun forrits á að koma upp upphafssýn sem sýnir nýjustu gögn um valin fyrirtæki, auk helstu frétta.

\item Titill: Margir gluggar\\
Lýsing: Forritið á að geta sýnt upplýsingar í mörgum gluggum.

\item Titill: Ábendingar\\
Lýsing: Forritið á að benda á fyrirtæki, vörur og gengjum sem virðast vera að ganga vel.

\item Titill: Vistun\\
Lýsing: Hægt á að vera að vista myndir af gröfum

\item Titill: Útprentun\\
Lýsing: Hægt á að vera að prenta út gröf.

\item Titill: Útflutningur\\
Lýsing: Hægt á að vera að flytja út upplýsingar í spreadsheet formatti.

\item Titill: Skipting eftir brönsum\\
Lýsing: Hægt á að vera að skipta fyrirtækjum eftir brönsum

\item Titill: Röðun\\
Lýsing: Hægt á að vera að raða listum af fyrirtækjum eftir ýmsum stærðum.

\item Titill: Aðvörun\\
Lýsing: Forritið á að vera hægt að stilla þannig að það gefi frá sér aðvörun þegar stærð fer út fyrir ákveðinn ramma.

\item Titill: Upplýsingar um fyrirtæki\\
Lýsing: Hægt á að vera að sjá fáanlegar upplýsingar um fyrirtæki á einum stað.

\item Titill: Borði\\
Lýsing: Forritið á að geta sýnt borða með völdum upplýsingum um valdar vísitölur, verslunarvörur og fyrirtæki.
\eenum

\end{document}
