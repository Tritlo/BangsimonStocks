\documentclass[a4paper,11pt]{article}

\usepackage[utf8]{inputenc}
\usepackage[T1]{fontenc}
\usepackage{a4,graphics,amsmath,amsfonts,amsbsy,amssymb,amsthm}
\usepackage{graphicx}
\usepackage{hyperref}
\usepackage{float}
\usepackage{listings}
\usepackage{enumerate}
\usepackage{comment}
\usepackage{accents}


\setlength{\parskip}{8pt plus 1pt minus 1pt}
%Verdur ad vera her, sumir pakkar dependa a thetta.
\usepackage[icelandic]{babel}

%viljum ekki númeraða kafla á dæmum

\setcounter{secnumdepth}{-1}

\usepackage[a4paper,margin=1in]{geometry}
\usepackage{fancyvrb}
%\usepackage{beramono} % or whatever monospaced font you wish to employ
\begin{document}
\pagestyle{empty} % don't need to display page numbers, right?

\title{Viðmótshönnun} \author{Finnur Jónasson \and Matthías Páll
  Gissurarson \and Ragnheiður Björk Halldórsdóttir \and Sólrún Halla
  Einarsdóttir}

\section{Upphafsgluggi}
\VerbatimInput{upphafsgluggi.asc}

\noindent
Inni í File er New og Exit.  Forritið opnar með upplýsingar um Google

\noindent
Ef ýtt er á New, þá kemur New gluggin.  Ef ýtt er á Exit, þá slökknar
á forritinu

\noindent
Inni í Plot er hægt að velja eiginleika sem sýndur er á grafinu, með
radio box.\\  Einnig er check box, sem hægt er að haka við og heitir
Moving Average,\\ og segir til um hvort maður eigi að taka moving
average eða ekki.

\noindent
Einnig er Customize, en þá kemur upp Plot gluggin.
\vfill
\section{New gluggi}


\VerbatimInput{newgluggi.asc}

\section{Plot glugginn}

\VerbatimInput{plotgluggi.asc}

\end{document}
